\scnsegmentheader{Описание реализации подсистемы взаимодействия с внешней средой с использованием сетевых языков}
\begin{scnsubstruct}

\scnheader{Реализация подсистемы взаимодействия с внешней средой с использованием сетевых языков}
\begin{scnrelfromlist}{компонент программной системы}
    \scnitem{Реализация подсистемы взаимодействия с внешней средой с использованием сетевых языков на основе языка JSON}
\end{scnrelfromlist}

\scntext{пояснение}{Взаимодействие программной модели sc-памяти с внешними ресурсами может осуществляться посредством специализированного программного интерфейса (API), однако этот вариант неудобен в большинстве случае, поскольку:
    \begin{scnitemize}
        \item поддерживается только для очень ограниченного набора языков программирования (С, С++);
        \item требует того, чтобы клиентское приложение, обращающееся к программной модели sc-памяти, фактически составляло с ней единое целое, таким образом исключается возможность построения распределенного коллектива ostis-систем;
        \item как следствие предыдущего пункта, исключается возможность параллельной работы с sc-памятью нескольких клиентских приложений.
    \end{scnitemize}
Для того, чтобы обеспечить возможность удаленного доступа к sc-памяти не учитывая при этом языки программирования, с помощью которых реализовано конкретное клиентское приложение, было принято решение о реализации возможности доступа к sc-памяти с использованием универсального языка, не зависящего от средств реализации того или иного компонента или системы. В качестве такого языка был разработан строковый язык на базе языка JSON.}

\scnstructheader{Описание подсистемы взаимодействия c sc-памятью на основе языка JSON}
\begin{scnsubstruct}

\scnheader{Реализация подсистемы взаимодействия c sc-памятью на основе языка JSON}
\scntext{пояснение}{Реализация подсистемы взаимодействия c sc-памятью на основе языка JSON позволяет ostis-системам взаимодействовать с системами из внешней среды на основе общепринятого транспортного формата передачи данных JSON и предоставляет API для доступа к sc-памяти платформы интерпретации sc-моделей.}
\begin{scnrelfromlist}{используемый язык программирования}
    \scnitem{C}
    \scnitem{C++}
    \scnitem{Python}
    \scnitem{TypeScript}
    \scnitem{C\#}
    \scnitem{Java}
\end{scnrelfromlist}
\begin{scnrelfromlist}{используемый язык}
    \scnitem{SC-JSON-код}
\end{scnrelfromlist}
\scnrelfrom{архитектура}{Клиент-серверная архитектура}
\scnrelto{реализация}{Подсистема взаимодействия с sc-памятью на основе языка JSON}
\begin{scnindent}
	\scnidtf{Подсистема взаимодействия с sc-памятью на основе формата JSON}
	\scnidtf{Подсистема взаимодействия с sc-памятью на основе транспортного формата передачи данных JSON}
	\scniselement{многократно используемый компонент ostis-систем}
	\scniselement{неатомарный многократно используемый компонент ostis-систем}
	\scniselement{зависимый многократно используемый компонент ostis-систем}
	\begin{scnrelfromlist}{автор}
	    \scnitem{Корончик Д. Н.}
	    \scnitem{Шункевич Д. В.}
	    \scnitem{Зотов Н. В.}
	    \scnitem{Загорский А. Г.}
	\end{scnrelfromlist}
	\scntext{пояснение}{Взаимодействие c sc-памятью обеспечивается с помощью передачи информации на \textit{\textbf{SC-JSON-коде}} и ведётся, с одной стороны, между сервером, являющегося частью ostis-системы, написанным на том же языке реализации этой ostis-системы и имеющим доступ к её sc-памяти, и с другой стороны множеством клиентом, которым известно о наличии сервера в пределах сети их использования.}
	\scntext{примечание}{Осмысленные фрагменты текстов \textit{\textbf{SC-JSON-кода}} представляют семантически корректную структуру сущностей и связей между ними.}
	\scntext{примечание}{С помощью подсистемы взаимодействия с sc-памятью на основе языка JSON можно взаимодействовать с ostis-системой на таком же множестве возможных операций, как и в случае, если бы взаимодействие происходило (непосредственно) напрямую, на том же языке реализации платформы интерпретации sc-моделей компьютерных систем. При этом результат работы отличается только скоростью обработки информации.}
	\begin{scnrelfromset}{декомпозиция программной системы}
		\scnitem{Серверная система на основе Websocket, обеспечивающая доступ к sc-памяти платформы интерпретации sc-моделей при помощи команд SC-JSON-кода}
		Множество клиентских систем, подключаемых и взаимодействующих с \textit{Серверной системой на основе Websocket, обеспечивающей доступ к sc-памяти платформы интерпретации sc-моделей при помощи команд SC-JSON-кода}\\
		\begin{scnindent}
			\begin{scnrelfromset}{декомпозиция программной системы}
				\scnitem{Клиентская система, подключаемая и взаимодействующая с \textit{SC-сервером}, реализованная на языке программирования Python}
				\scnitem{Клиентская система, подключаемая и взаимодействующая с \textit{SC-сервером}, реализованная на языке программирования TypeScript}
				\scnitem{Клиентская система, подключаемая и взаимодействующая с \textit{SC-сервером}, реализованная на языке программирования C\#}
				\scnitem{Клиентская система, подключаемая и взаимодействующая с \textit{SC-сервером}, реализованная на языке программирования Java}
			\end{scnrelfromset}
		\end{scnindent}
    \end{scnrelfromset}
\end{scnindent}

\scnheader{SC-JSON-код}
\scnidtf{Semantic JSON-code}
\scnidtf{Semantic JavaScript Object Notation code}
\scnidtf{Язык внешнего смыслового представления знаний для взаимодействия с ostis-системами на основе языка JSON}
\scnidtf{Метаязык, являющийся подмножеством языка JSON и обеспечивающий внешнее представление и структуризацию \textit{sc-текстов}, используемых ostis-системой в процессе своего функционирования и взаимодействия со внешней средой.}
\scntext{часто используемый неосновной внешний идентификатор sc-элемента}{sc-json-текст}
\begin{scnindent}
    \scniselement{имя нарицательное}
\end{scnindent}
\scniselement{абстрактный язык}
\scniselement{линейный язык}
\scnsubset{JSON}
\begin{scnrelfromlist}{автор}
	\scnitem{Зотов Н. В.}
	\scnitem{Корончик Д. Н.}
\end{scnrelfromlist}
\begin{scnrelfromvector}{принципы, лежащие в основе}
    \scnitem{Тексты, описываемые на языке внешнего представления знаний \textit{\textbf{SC-JSON-код}} представляют собой линейную структуру, представляемую в виде линейного строкового текста и состоящую из набора корректных осмысленных команд, записанных в виде \textit{sc-json-пар} вида \{отношение: объект\}, где отношением выступает знак квазибинарного отношения, состоящего из пар вида \{субъект: объект\}, где объектом выступает знак, обозначаемый предложением, включающее такие пары, а субъектом - sc-json-объекты: sc-json-литерал, sc-json-списки sc-json-объектов, sc-json-предложения, состоящие из sc-json-списков sc-json-объектов.}
    \scnitem{Тексты \textit{\textbf{SC-JSON-кода}} представляют собой sc-json-команды. Каждая команда представляет собой json-объект, в котором указываются уникальный идентификатор команды, тип этой команды и ее аргументы. C каждой командой ассоциируется ответ на эту команду. Ответ на команду представляет собой команду, в котором указываются идентификатор команды, ее статус (выполнена успешно/безуспешно) и результаты. Структура аргументов и результатов команды определяется типом команды. Для каждого ответа существует запрос.}
\end{scnrelfromvector}
\begin{scnrelfromlist}{достоинство}
    \scnitem{\scnfilelong{Язык JSON является общепринятым открытым форматом, для работы с которым существует большое количество библиотек для популярных языков программирования. Это, в свою очередь, упрощает реализацию клиента и сервера для протокола, построенного на базе \textit{\textbf{SC-JSON-код}}.}}
    \scnitem{\scnfilelong{Реализация подсистемы взаимодействия со внешней средой на базе \textit{\textbf{SC-JSON-код}} не накладывает принципиальных ограничений на объем (длину) каждой команды, в отличие от других бинарных протоколов. Таким образом, появляется возможность использования неатомарных команд, позволяющих, например, за один акт пересылки такой команды по сети создать сразу несколько sc-элементов. Важными примерами таких команд являются \textit{команда создания sc-конструкции, изоморфной заданному sc-шаблону}, и \textit{команда поиска sc-конструкций, изоморфных заданному sc-шаблону}.}}
\end{scnrelfromlist}
\scntext{примечание}{Можно сказать, что язык на базе JSON является следующим шагом на пути к созданию мощного и универсального языка запросов, аналогичного языку SQL для реляционных баз данных и предназначенному для работы с sc-памятью. Следующий шагом станет реализация такого протокола на основе одного из стандартов внешнего отображения sc-конструкций, например, \textit{SCs-кода}, что, в свою очередь, позволит передавать в качестве команд целые программы обработки sc-конструкций, например на языке SCP.}

\scnstructheader{Синтаксис SC-JSON-кода}
\begin{scnsubstruct}
\scnheader{Синтаксис SC-JSON-кода}
\scntext{примечание}{\textit{Синтаксис SC-JSON-кода} задается: (1) \textit{Алфавитом SC-JSON-кода}, (2) Грамматикой SC-JSON-кода}
\scnrelto{синтаксис}{SC-JSON-код}

\scnstructheader{Синтаксическая классификация элементов SC-JSON-кода}
\begin{scnsubstruct}

\scnstructheader{SC-JSON-код}
\scnrelto{семейство подмножеств}{sc-json-предложение}
\begin{scnindent}
	\scnsubset{json-список json-пар}
	\scnrelto{семейство подмножеств}{sc-json-пара*}
	\begin{scnindent}
		\begin{scnreltovector}{декартово произведение}
		    \scnitem{sc-json-строка}
			\scnitem{sc-json-объект}
			\begin{scnindent}
				\begin{scnrelfromset}{разбиение}
					\scnitem{sc-json-cписок}
					\scnitem{sc-json-пара}
					\scnitem{sc-json-литерал}
					\begin{scnindent}
						\begin{scnrelfromset}{разбиение}
							\scnitem{sc-json-строка}
							\scnitem{sc-json-число}
					    \end{scnrelfromset}
					\end{scnindent}
				\end{scnrelfromset}
		    \end{scnindent}
		\end{scnreltovector}
	\end{scnindent}
	\begin{scnrelfromset}{разбиение}
		\scnitem{команда на SC-JSON-коде}
		\scnitem{ответ на команду на SC-JSON-коде}
	\end{scnrelfromset}
\end{scnindent}

\end{scnsubstruct}
\scnsourcecomment{Завершили представление \textit{Синтаксической классификации элементов SC-JSON-кода}}

\scnheader{Алфавит SC-JSON-кода\scnsupergroupsign}
\scnidtf{Множество всех возможных символов в SC-JSON-коде}
\scntext{пояснение}{Поскольку SC-JSON-код является линейным строковым языком представления знаний, то его алфавит включает объединение алфавитов всех языков, тексты на которых могут представлять внешние идентификаторы и/или содержимое файлов ostis-системы, множество всех цифр и множество всех других специальных символов.}
\scnrelto{алфавит}{SC-JSON-код}
\scntext{примечание}{Последовательности знаков алфавита могут образовывать sc-json ключевые слова, sc-json-пары, sc-json-предложения из sc-json-пар и sc-json-тексты из sc-json-предложений.}

\scnheader{SC-JSON-код}
\begin{scnrelfromlist}{синтаксические правила}
    \scnitem{\scnfilelong{Каждое правило \textit{Грамматики SC-JSON-кода} описывает корректный с точки зрения \textit{Синтаксиса SC-JSON-кода} порядок sc-json-объектов в sc-json-предложении. Совокупность правил \textit{Грамматики SC-JSON-кода} описывает корректный с точки зрения \textit{Синтаксиса SC-JSON-кода} порядок sc-json-предложений в sc-json-тексте.}}
    \scnitem{\scnfilelong{Каждое sc-json-предложение является sc-json-списком, состоящим из sc-json-пар и представляет собой команду или ответ на эту команду.}}
    \scnitem{\scnfilelong{Каждое \textit{команда (ответ на команду) на SC-JSON-коде} состоит из заголовка, включающего sc-json-пары описания самой команды (ответа на команду), и сообщения, различного для каждого класса команд (ответов на команды). Сообщение {команды (ответа на команду) на SC-JSON-коде} обычно представляет собой список sc-json-объектов и может не ограничиваться по мощности.}}
    \scnitem{\scnfilelong{Каждая sc-json-пара состоит из двух элементов: ключевого слова и некоторого другого sc-json-объекта, ассоциируемого с этим ключевым словом. Набор ключевых слов в sc-json-парах определяется конкретным классом {команд (ответов на команды) на SC-JSON-коде}. Sc-json-пара начинается знаком открывающейся фигурной скобки \scnqq{\{} и заканчивается знаком закрывающейся фигурной скобки \scnqq{\}}. Ключевое слово и sc-json-объект, ассоциируемый с ним, разделяются при помощи знака двоеточия \scnqq{:}.}}
    \scnitem{\scnfilelong{Sc-json-строки, записанные в sc-json-текстах, начинаются и заканчиваются знаком двух ковычек \textquotedblleft.}}
    \scnitem{\scnfilelong{Sc-json-списки, состоящие не из sc-json-пар, начинаются знаком открывающейся квадратной скобки \scnqq{[} и заканчиваются знаком закрывающейся квадратной скобки \scnqq{]}. Sc-json-объекты в sc-json-списках разделяются запятыми \scnqq{,}.}}
\end{scnrelfromlist}

\scnheader{Грамматика SC-JSON-кода}
\scnidtf{Множество всех возможных правил, используемых при построении команд и ответов на них на SC-JSON-коде}
\scntext{пояснение}{Каждой команде \textit{SC-JSON-кода} однозначно соответствует правило грамматики \textit{SC-JSON-кода}.}
\scnrelto{грамматика}{SC-JSON-код}
\scntext{пояснение}{Правила \textit{Грамматики SC-JSON-кода} позволяют правильно составить команду на SC-JSON-коде. Каждое правило грамматики \textit{SC-JSON-кода} представляется в виде правила на \textit{Языке описания грамматик ANTLR} и его интерпретации на естественном языке.}

\scnhaselementrole{ключевой sc-элемент}{Правило, задающее синтаксис \textit{команд на SC-JSON-коде}}
\begin{scnindent}
	\scnrelboth{семантическая эквивалентность}{\scnfileimage[20em]{sc-server/images/command}}
	\begin{scnindent}
		\scniselement{Язык описания грамматики языков ANTLR}
		\scntext{интерпретация}{Класс \textit{команд на SC-JSON-коде} включает \textit{команду создания sc-элементов}, \textit{команду получения соответствующих типов sc-элементов}, \textit{команду удаления sc-элементов}, \textit{команду обработки ключевых sc-элементов}, \textit{команду обработки содержимого файлов ostis-системы}, \textit{команду поиска sc-конструкций, изоморфных заданному sc-шаблону}, \textit{команду генерации sc-конструкции, изоморфной заданному sc-шаблону}, и \textit{команду обработки sc-событий}. В \textit{команду на SC-JSON-коде} включаются идентификатор этой команды, тип и сообщение.}
	\end{scnindent}
	\scnrelto{синтаксическое правило}{команда на SC-JSON-коде}
\end{scnindent}

\scnhaselementrole{ключевой sc-элемент}{Правило, задающее синтаксис \textit{ответа на команду на SC-JSON-коде}}
\begin{scnindent}
	\scnrelboth{семантическая эквивалентность}{\scnfileimage[20em]{sc-server/images/command_answer.png}}
	\begin{scnindent}
		\scniselement{Язык описания грамматики языков ANTLR}
		\scntext{интерпретация}{Класс \textit{ответов на команды на SC-JSON-коде} включает \textit{ответ на команду создания sc-элементов}, \textit{ответ на команду получения соответствующих типов sc-элементов}, \textit{ответ на команду удаления sc-элементов}, \textit{ответ на команду обработки ключевых sc-элементов}, \textit{ответ на команду обработки содержимого файлов ostis-системы}, \textit{ответ на команду поиска sc-конструкций, изоморфных заданному sc-шаблону}, \textit{ответ на команду генерации sc-конструкции, изоморфной заданному sc-шаблону}, и \textit{ответ на команду обработки sc-событий}. В \textit{ответ на команду на SC-JSON-коде} включаются идентификатор соответствующей команды, статус обработки ответа и ответное сообщение.}
	\end{scnindent}
	\scnrelto{синтаксическое правило}{ответ на команду на SC-JSON-коде}
\end{scnindent}

\scnhaselement{Правило, задающее синтаксис \textit{команды создания sc-элементов}}
\begin{scnindent}
	\scnrelboth{семантическая эквивалентность}{\scnfileimage[50em]{sc-server/images/create_elements_command.pdf}}
	\begin{scnindent}
		\scniselement{Язык описания грамматики языков ANTLR}
		\scntext{интерпретация}{В сообщении \textit{команды создания sc-элементов} представляется список описаний создаваемых sc-элементов. Такими sc-элементами могут быть sc-узел, sc-дуга, sc-ребро или файл ostis-системы. Тип sc-элемента указывается в паре с ключевым словом \scnqq{el}: для sc-узла sc-json-тип элемент представляется как \scnqq{node}, для sc-дуги и sc-ребра - \scnqq{edge}, для файла ostis-системы - \scnqq{link}. Метки типов sc-элементов уточняются в соответствующих им описаниях в сообщении команды в паре с ключевым словом \scnqq{type}. Если создаваемым sc-элементом является файл ostis-системы, то дополнительно указывается содержимое этого файла ostis-системы в паре с ключевым словом \scnqq{content}, если создаваемым sc-элементом является sc-дуга или sc-ребро, то указываются описания sc-элементов, из которых они выходят, и sc-элементов, в которые они входят. Описание таких sc-элементов состоят из двух пар: первая пара указывает на способ ассоциации с sc-элементом и представляется как \scnqq{addr} или \scnqq{idtf} или \scnqq{ref} в паре с ключевым словом \scnqq{type}, вторая пара - то, по чему происходит ассоциация с этим sc-элементом: его хэшу, системному идентификатору или номеру в массиве создаваемых sc-элементов - в паре с ключевым словом \scnqq{value}.}
	\end{scnindent}
	\scnrelto{синтаксическое правило}{команда создания sc-элементов}
\end{scnindent}

\scnhaselement{Правило, задающее синтаксис \textit{ответа на команду создания sc-элементов}}
\begin{scnindent}
	\scnrelboth{семантическая эквивалентность}{\scnfileimage[20em]{sc-server/images/create_elements_command_answer.png}}
	\begin{scnindent}
		\scniselement{Язык описания грамматики языков ANTLR}
		\scntext{интерпретация}{Сообщением \textit{ответа на команду создания sc-элементов} является список хэшей созданных sc-элементов, соответствующих описаниям \textit{команды создания sc-элементов} со статусом 1, в случае успешной обработки команды.}
	\end{scnindent}
	\scnrelto{синтаксическое правило}{ответ на команду создания sc-элементов}
\end{scnindent}

\scnhaselement{Правило, задающее синтаксис \textit{команды создания sc-элементов по фрагменту SCs-текста}}
\begin{scnindent}
\scnrelboth{семантическая эквивалентность}{\scnfileimage[20em]{sc-server/images/create_elements_by_scs_command.png}}
\begin{scnindent}
\scniselement{Язык описания грамматики языков ANTLR}
\scntext{интерпретация}{В списке описаний создаваемых sc-элементов сообщения этой команды вместо описания создаваемого отдельного sc-элемента указывается фрагмент SCs-текста.}
\end{scnindent}
\scnrelto{синтаксическое правило}{команда создания sc-элементов по фрагменту SCs-текста}
\end{scnindent}

\scnhaselement{Правило, задающее синтаксис \textit{ответа на команду создания sc-элементов по фрагменту SCs-текста}}
\begin{scnindent}
\scnrelboth{семантическая эквивалентность}{\scnfileimage[20em]{sc-server/images/create_elements_by_scs_command_answer.png}}
\begin{scnindent}
\scniselement{Язык описания грамматики языков ANTLR}
\scntext{интерпретация}{Сообщением \textit{ответа на команду создания sc-элементов} является список результатов обработки переданных SCs-текстов. Нулевой статус говорит о том, что обработка соотвествующего SCs-текста завершилась безуспешно.}
\end{scnindent}
\scnrelto{синтаксическое правило}{ответ на команду создания sc-элементов по фрагменту SCs-текста}
\end{scnindent}

\scnhaselement{Правило, задающее синтаксис \textit{команды получения соответствующих типов sc-элементов}}
\begin{scnindent}
	\scnrelboth{семантическая эквивалентность}{\scnfileimage[15em]{sc-server/images/check_elements_command.png}}
	\begin{scnindent}
		\scniselement{Язык описания грамматики языков ANTLR}
		\scntext{интерпретация}{Сообщением \textit{команды получения соответствующих типов sc-элементов} является списком хэшей sc-элементов, у которых необходимо получить синтаксические типы.}
	\end{scnindent}
	\scnrelto{синтаксическое правило}{команда получения соответствующих типов sc-элементов}
\end{scnindent}

\scnhaselement{Правило, задающее синтаксис \textit{ответа на команду получения соответствующих типов sc-элементов}}
\begin{scnindent}
	\scnrelboth{семантическая эквивалентность}{\scnfileimage[20em]{sc-server/images/check_elements_command_answer.png}}
	\begin{scnindent}
		\scniselement{Язык описания грамматики языков ANTLR}
		\scntext{интерпретация}{Сообщением \textit{ответа на команду получения соответствующих типов sc-элементов} является список типов проверенных sc-элементов, соответствующих описаниям \textit{команды получения соответствующих типов sc-элементов} со статусом 1, в случае успешной обработки команды.}
	\end{scnindent}
	\scnrelto{синтаксическое правило}{ответ на команду получения соответствующих типов sc-элементов}
\end{scnindent}

\scnhaselement{Правило, задающее синтаксис \textit{команды удаления sc-элементов}}
\begin{scnindent}
	\scnrelboth{семантическая эквивалентность}{\scnfileimage[15em]{sc-server/images/delete_elements_command.png}}
	\begin{scnindent}
		\scniselement{Язык описания грамматики языков ANTLR}
		\scntext{интерпретация}{Сообщением \textit{команды удаления sc-элементов} является список хэшей sc-элементов, которые необходимо удалить из sc-памяти.}
	\end{scnindent}
	\scnrelto{синтаксическое правило}{команда удаления sc-элементов}
\end{scnindent}

\scnhaselement{Правило, задающее синтаксис \textit{ответа на команду удаления sc-элементов}}
\begin{scnindent}
	\scnrelboth{семантическая эквивалентность}{\scnfileimage[20em]{sc-server/images/delete_elements_command_answer.png}}
	\begin{scnindent}
		\scniselement{Язык описания грамматики языков ANTLR}
		\scntext{интерпретация}{Сообщение \textit{ответа на команду удаления sc-элементов} является пустым со статусом 1, в случае успешной обработки команды.}
	\end{scnindent}
	\scnrelto{синтаксическое правило}{ответ на команду удаления sc-элементов}
\end{scnindent}

\scnhaselement{Правило, задающее синтаксис \textit{команды обработки ключевых sc-элементов}}
\begin{scnindent}
	\scnrelboth{семантическая эквивалентность}{\scnfileimage[20em]{sc-server/images/handle_keynodes_command.png}}
	\begin{scnindent}
	\scniselement{Язык описания грамматики языков ANTLR}
	\scntext{интерпретация}{Сообщение \textit{команды обработки ключевых sc-элементов} может включать описание ключевых sc-элементов, которые необходимо найти и/или разрешить по их идентификаторам. Такое деление осуществляется с помощью подкоманд, содержащихся в сообщении команды. Идентификаторами подкоманд могут быть "find"{} и "resolve"{} соответственно, стоящие в паре с ключевым словом "command". Описание искомого sc-элемента команды "find"{} включает системный идентификатор sc-элемента, по которому необходимо найти этот sc-элемент, стоящий в паре с ключевым словом "idtf". Описание разрешаемого sc-элемента команды "resolve"{} включает системный идентификатор sc-элемента, по которому необходимо найти этот sc-элемент, либо в случае безуспешного поиска создать sc-элемент некоторого типа, указанного в его описании в паре с ключевым словом "elType"{}, и установить для него системный идентификатор, по которому была произведена попытка найти другой sc-элемент.}
	\end{scnindent}
	\scnrelto{синтаксическое правило}{команда обработки ключевых sc-элементов}
\end{scnindent}

\scnhaselement{Правило, задающее синтаксис \textit{ответа на команду обработки ключевых sc-элементов}}
\begin{scnindent}
	\scnrelboth{семантическая эквивалентность}{\scnfileimage[20em]{sc-server/images/handle_keynodes_command_answer.png}}
	\begin{scnindent}
	\scniselement{Язык описания грамматики языков ANTLR}
	\scntext{интерпретация}{Сообщением \textit{ответа на команду обработки ключевых sc-элементов} является список хэшей sc-элементов, соответствующих описаниям \textit{команды обработки ключевых sc-элементов} со статусом 1, в случае успешной обработки команды.}
	\end{scnindent}
	\scnrelto{синтаксическое правило}{ответ на команду обработки ключевых sc-элементов}
\end{scnindent}

\scnhaselement{Правило, задающее синтаксис \textit{команды обработки содержимого файлов ostis-системы}}
\begin{scnindent}
	\scnrelboth{семантическая эквивалентность}{\scnfileimage[30em]{sc-server/images/handle_link_contents_command.png}}
	\begin{scnindent}
		\scniselement{Язык описания грамматики языков ANTLR}
		\scntext{интерпретация}{Сообщение \textit{команды обработки содержимого файлов ostis-системы} может включать описание ключевых файлов ostis-системы, которые необходимо найти по их содержимому или части этого содержимого, для которых необходимо установить содержимое разрешить и/или у которых необходимо получить содержимое. Как и в \textit{Правиле, задающее синтаксис команды обработки ключевых sc-элементов} деление осуществляется с помощью подкоманд, содержащихся в сообщении команды. Идентификаторами подкоманд могут быть \scnqq{find}, \scnqq{find\_by\_substr}, \scnqq{set} и \scnqq{get} соответственно, стоящие в паре с ключевым словом \scnqq{command}. В описаниях команд \scnqq{set} и \scnqq{get} указывается хэш файла ostis-системы в паре с ключевым словом \scnqq{addr}. В описаниях команд \scnqq{set}, \scnqq{find} и \scnqq{find\_by\_substr} указывается содержимое файла ostis-системы в паре с ключевым словом \scnqq{data}. Дополнительно в описании подкоманды \scnqq{set} указывается тип устанавливаемого содержимого файла ostis-системы.}
	\end{scnindent}
	\scnrelto{синтаксическое правило}{команда обработки содержимого файлов ostis-системы}
\end{scnindent}

\scnhaselement{Правило, задающее синтаксис \textit{ответа на команду обработки содержимого файлов ostis-системы}}
\begin{scnindent}
	\scnrelboth{семантическая эквивалентность}{\scnfileimage[30em]{sc-server/images/handle_link_contents_command_answer}}
	\begin{scnindent}
		\scniselement{Язык описания грамматики языков ANTLR}
		\scntext{интерпретация}{Сообщением \textit{ответа на команду обработки содержимого файлов ostis-системы} является список, состоящий из булевого результата установки содержимого в файл ostis-системы и/или найденных файлов ostis-системы по их содержимому и/или описания полученного содержимого файлов ostis-системы, соответствующих описаниям \textit{команды обработки содержимого файлов ostis-системы} со статусом 1, в случае успешной обработки команды.}
	\end{scnindent}
	\scnrelto{синтаксическое правило}{ответ на команду обработки содержимого файлов ostis-системы}
\end{scnindent}

\scnhaselement{Правило, задающее синтаксис \textit{команды поиска sc-конструкций, изоморфных заданному sc-шаблону}}
\begin{scnindent}
	\scnrelboth{семантическая эквивалентность}{\scnfileimage[20em]{sc-server/images/search_template_command.png}}
	\begin{scnindent}
		\scniselement{Язык описания грамматики языков ANTLR}
		\scnsubset{Правило, задающее синтаксис сообщения \textit{команды поиска sc-конструкций, изоморфных заданному sc-шаблону}, и \textit{команды генерации sc-конструкции, изоморфной заданному sc-шаблону}}
		\scntext{интерпретация}{\textit{Правило, задающее синтаксис команды поиска sc-конструкций, изоморфных заданному sc-шаблону} включает \textit{Правило, задающее синтаксис сообщения \textit{команды поиска sc-конструкций, изоморфных заданному sc-шаблону,} и \textit{команды генерации sc-конструкции, изоморфной заданному sc-шаблону}} и описывает команду поиска sc-конструкций по сформированному этой командой sc-шаблону (см. \textit{Правило, задающее синтаксис сообщения \textit{команды поиска sc-конструкций, изоморфных заданному sc-шаблону,} и \textit{команды генерации sc-конструкции, изоморфной заданному sc-шаблону}}).}
	\end{scnindent}
	\scnrelto{синтаксическое правило}{команда поиска sc-конструкций, изоморфных заданному sc-шаблону}
\end{scnindent}

\scnhaselement{Правило, задающее синтаксис \textit{ответа на команду поиска sc-конструкций, изоморфных заданному sc-шаблону}}
\begin{scnindent}
	\scnrelboth{семантическая эквивалентность}{\scnfileimage[15em]{sc-server/images/search_template_command_answer.png}}
	\begin{scnindent}
		\scniselement{Язык описания грамматики языков ANTLR}
		\scntext{интерпретация}{Сообщение \textit{ответа на команду поиска sc-конструкций, изоморфных заданному sc-шаблону} состоит из списка найденных sc-конструкций и отображения псевдонимов sc-элементов на их позиции в тройках sc-шаблона. Ответ имеет статус 1, в случае успешной обработки команды.}
	\end{scnindent}
	\scnrelto{синтаксическое правило}{ответ на команду поиска sc-конструкций, изоморфных заданному sc-шаблону}
\end{scnindent}

\scnhaselement{Правило, задающее синтаксис \textit{команды создания sc-конструкции, изоморфной заданному sc-шаблону}}
\begin{scnindent}
	\scnrelboth{семантическая эквивалентность}{\scnfileimage[20em]{sc-server/images/generate_template_command.png}}
	\begin{scnindent}
		\scniselement{Язык описания грамматики языков ANTLR}
		\scnsubset{Правило, задающее синтаксис сообщения \textit{команды поиска sc-конструкций, изоморфных заданному sc-шаблону,} и \textit{команды генерации sc-конструкции, изоморфной заданному sc-шаблону}}
		\scntext{интерпретация}{\textit{Правило, задающее синтаксис команды создания sc-конструкции, изоморфной заданному sc-шаблону} включает \textit{Правило, задающее синтаксис сообщения \textit{команды поиска sc-конструкций, изоморфных заданному sc-шаблону,} и \textit{команды генерации sc-конструкции, изоморфной заданному sc-шаблону}} и описывает команду создания sc-конструкции по сформированному этой командой sc-шаблону (см. \textit{Правило, задающее синтаксис сообщения \textit{команды поиска sc-конструкций, изоморфных заданному sc-шаблону,} и \textit{команды генерации sc-конструкции, изоморфной заданному sc-шаблону}}).}
	\end{scnindent}
	\scnrelto{синтаксическое правило}{команда создания sc-конструкции, изоморфной заданному sc-шаблону}
\end{scnindent}

\scnhaselement{Правило, задающее синтаксис \textit{ответа на команду создания sc-конструкции, изоморфной заданному sc-шаблону}}
\begin{scnindent}
	\scnrelboth{семантическая эквивалентность}{\scnfileimage[15em]{sc-server/images/generate_template_command_answer.png}}
	\begin{scnindent}
		\scniselement{Язык описания грамматики языков ANTLR}
		\scntext{интерпретация}{Сообщение \textit{ответа на команду создания sc-конструкции, изоморфной заданному sc-шаблону} состоит из списка найденных sc-конструкций и отображения псевдонимов sc-элементов на их позиции в тройках sc-шаблона. Ответ имеет статус 1, в случае успешной обработки команды.}
	\end{scnindent}
	\scnrelto{синтаксическое правило}{ответ на команду создания sc-конструкции, изоморфной заданному sc-шаблону}
\end{scnindent}

\scnhaselement{Правило, задающее синтаксис сообщения \textit{команды поиска sc-конструкций, изоморфных заданному sc-шаблону,} и \textit{команды создания sc-конструкции, изоморфной заданному sc-шаблону}}
\begin{scnindent}
	\scnrelboth{семантическая эквивалентность}{\scnfileimage[24em]{sc-server/images/template_message_command.png}}
	\begin{scnindent}
		\scniselement{Язык описания грамматики языков ANTLR}
		\scntext{интерпретация}{Сообщения \textit{команды поиска sc-конструкций, изоморфных заданному sc-шаблону,} и \textit{команды создания sc-конструкции, изоморфной заданному sc-шаблону} включают описание троек, необходимых для создания sc-шаблона поиска или генерации изоморфных sc-конструкций. Описание каждой тройки sc-шаблона включает описание sc-элементов этой тройки. Описания первого и третьего sc-элементов тройки должны всегда содержать хэш или тип в паре с ключевым словом \scnqq{value}. Если выбран тип, то в паре с ключевым словом \scnqq{type} указывается \scnqq{type}, если - хэш, то - \scnqq{addr}. Вторым sc-элементом тройки должна быть дуга, для которой всегда указывается тип в паре с ключевым словом \scnqq{value}. Для каждого sc-элемента тройки может указываться псевдоним в паре с \scnqq{alias}. Псевдоним представляет собой строку и может быть использован для ассоциации с sc-элементами в других тройках sc-шаблона, либо ассоциации со значениями переменных sc-шаблона, которые указываются в списке под ключевым словом \scnqq{params} и могут представлять собой либо хэш sc-элемента, либо его системный идентификатор. Таким образом, в некоторых случаях может отсутствовать необходимость указания хэша или типа sc-элемента. Также вместо списка описаний троек sc-шаблона, может указываться хэш или системный идентификатор sc-структуры, хранящейся в sc-памяти. хэш и системный идентификатор указываются в паре с ключевым словом \scnqq{value}.}
	\end{scnindent}
\end{scnindent}

\scnhaselement{Правило, задающее синтаксис \textit{команды обработки sc-событий}}
\begin{scnindent}
	\scnrelboth{семантическая эквивалентность}{\scnfileimage[15em]{sc-server/images/handle_events_command.png}}
	\begin{scnindent}
		\scniselement{Язык описания грамматики языков ANTLR}
		\scntext{интерпретация}{Сообщение \textit{команды обработки sc-событий} может включать описание sc-элементов, по котором необходимо зарегистрировать или разрегистрировать sc-события. Идентификаторами подкоманд в описании команды могут быть \scnqq{create} и \scnqq{delete} соответственно, стоящие в паре с ключевым словом \scnqq{command}. Описание команды регистрации sс-cобытий \scnqq{create} представляет собой список описаний типов sc-событий и sc-элементов, по которым необходимо зарегистрировать sc-события. Описания sc-элементов включают хэши этих sc-элементов в парах с ключевым словом \scnqq{addr}. Описание команды разрегистрации sc-событий представляет собой список позиций sc-событий в очереди sc-событий, которые необходимо удалить из этой очереди sc-событий.}
	\end{scnindent}
	\scnsuperset{Правило, задающее синтаксис типов sc-событий}
	\begin{scnindent}
		\scnrelboth{семантическая эквивалентность}{\scnfileimage[12em]{sc-server/images/sc_event_types.png}}
		\begin{scnindent}
			\scniselement{Язык описания грамматики языков ANTLR}
			\scntext{интерпретация}{Sc-событиями могут быть \textit{sc-события добавления выходящей дуги из sc-элемента (add\_outgoing\_edge)}, \textit{sc-события добавления входящей дуги в sc-элемент\\ (add\_ingoing\_edge)}, \textit{sc-события удаления выходящей дуги из sc-элемента\\ (remove\_outgoing\_edge)}, \textit{sc-события удаления входящей дуги в sc-элемент\\(remove\_ingoing\_edge)}, \textit{sc-события изменения содержимого файла ostis-системы\\ (content\_change)} и \textit{sc-события удаления sc-элемента (delete\_element)}.}
		\end{scnindent}
	\end{scnindent}
	\scnrelto{синтаксическое правило}{команда обработки sc-событий}
\end{scnindent}

\scnhaselement{Правило, задающее синтаксис \textit{ответа на команду обработки sc-событий}}
\begin{scnindent}
	\scnrelboth{семантическая эквивалентность}{\scnfileimage[15em]{sc-server/images/handle_events_command_answer.png}}
	\begin{scnindent}
		\scniselement{Язык описания грамматики языков ANTLR}
		\scntext{интерпретация}{Сообщение \textit{ответа на команду обработки sc-событий} состоит из позиций зарегистрированных sc-событий в очереди. Успешным результатом \textit{команды обработки sc-событий} является статус 1.}
	\end{scnindent}
	\scnrelto{синтаксическое правило}{ответ на команду обработки sc-событий}
\end{scnindent}

\scnhaselement{Правило, задающее синтаксис \textit{ответа инициализации sc-события}}
\begin{scnindent}
    \scnrelboth{семантическая эквивалентность}{\scnfileimage[15em]{sc-server/images/init_event_command_answer.png}}
    \begin{scnindent}
        \scniselement{Язык описания грамматики языков ANTLR}
        \scntext{интерпретация}{\textit{Ответ инициализации sc-события} возникает тогда и только тогда, когда в sc-памяти инициализируется соответствующее sc-событие. \textit{Ответ инициализации sc-события} всегда отсылается той клиентской системе, которая зарегистрировала это sc-событие. В сообщение \textit{ответа инициализации sc-события} включаются хэши тех sc-элементов, которые связаны с зарегистрированным sc-событием. Таким образом, если было зарегистрировано sc-событие выходящей sc-дуги, то в списке сообщения \textit{ответа инициализации sc-события} будут находится хэши трёх sc-элементов: хэш sc-элемента, который был подписан на sc-событие, хэш добавленной выходящей из него sc-дуги и хэш sc-элемента, являющегося концом этой sc-дуги.}
    \end{scnindent}
    \scnrelto{синтаксическое правило}{ответ инициализации sc-события}
\end{scnindent}

\scnhaselement{Правило, задающее синтаксис \textit{синтаксических типов sc-элементов}}
\begin{scnindent}
    \scnrelboth{семантическая эквивалентность}{\scnfileimage[9em]{sc-server/images/sc_addr_types.png}}
    \begin{scnindent}
    \scniselement{Язык описания грамматики языков ANTLR}
        \scntext{интерпретация}{\textit{Правило, задающее синтаксис синтаксических типов sc-элементов} включает \textit{Правило, задающее синтаксис синтаксических типов sc-узлов}, \textit{Правило, задающее синтаксис синтаксических типов sc-дуг}, \textit{Правило, задающее синтаксис синтаксических типов файлов ostis-системы}. Синтаксические типы sc-элементов представляются в виде целого числа и соответствуют программным синтаксическим типам, представляемым в sc-памяти.}
    \end{scnindent}
    \scntext{примечание}{На данный момент форма представления синтаксического типа sc-элемента зависит от того, как располагаются биты в маске sc-элемента. Следующим шагом в развитии \textit{SC-JSON-кода} и его грамматики могли быть стать устранение такой зависимости и переход к представлению синтаксических типов в виде строковых литералов, интерпретируемых \textit{Серверной системы на основе Websocket, обеспечивающей доступ к sc-памяти платформы интерпретации sc-моделей при помощи команд SC-JSON-кода}.}
\end{scnindent}

\scnhaselement{Правило, задающее синтаксис \textit{синтаксических типов sc-узлов}}
\begin{scnindent}
    \scnrelboth{семантическая эквивалентность}{\scnfileimage[18em]{sc-server/images/sc_node_types.png}}
        \begin{scnindent}
            \scniselement{Язык описания грамматики языков ANTLR}
            \scntext{интерпретация}{\textit{Правило, задающее синтаксис синтаксических типов sc-узлов} описывает возможные синтаксические типы sc-узлов, интерпретируемые на стороне \textit{Серверной системы на основе Websocket, обеспечивающей доступ к sc-памяти платформы интерпретации sc-моделей при помощи команд SC-JSON-кода}.}
    \end{scnindent}
\end{scnindent}

\scnhaselement{Правило, задающее синтаксис \textit{синтаксических типов sc-дуг}}
\begin{scnindent}
    \scnrelboth{семантическая эквивалентность}{\scnfileimage[20em]{sc-server/images/sc_edge_types.png}}
    \begin{scnindent}
        \scniselement{Язык описания грамматики языков ANTLR}
        \scntext{интерпретация}{\textit{Правило, задающее синтаксис синтаксических типов sc-дуг} описывает возможные синтаксические типы sc-дуг, в том числе и sc-рёбер, интерпретируемые на стороне \textit{Серверной системы на основе Websocket, обеспечивающей доступ к sc-памяти платформы интерпретации sc-моделей при помощи команд SC-JSON-кода}.}
    \end{scnindent}
\end{scnindent}

\scnhaselement{Правило, задающее синтаксис \textit{синтаксических типов файлов ostis-системы}}
\begin{scnindent}
    \scnrelboth{семантическая эквивалентность}{\scnfileimage[12em]{sc-server/images/sc_link_types.png}}
    \begin{scnindent}
        \scniselement{Язык описания грамматики языков ANTLR}
        \scntext{интерпретация}{\textit{Правило, задающее синтаксис синтаксических типов файлов ostis-системы} описывает возможные синтаксические типы файлов ostis-системы, интерпретируемые на стороне \textit{Серверной системы на основе Websocket, обеспечивающей доступ к sc-памяти платформы интерпретации sc-моделей при помощи команд SC-JSON-кода}.}
    \end{scnindent}
\end{scnindent}

\scnheader{команда на SC-JSON-коде}
\scnidtf{sc-json-code command}
\scnsubset{SC-JSON-код}
\scntext{примечание}{Множество \textit{команд на SC-JSON-коде} легко расширяемо засчёт гибкости и функциональности языка JSON.}

\scnheader{ответ на команду на SC-JSON-коде}
\scnidtf{sc-json-code command answer}
\scnsubset{SC-JSON-код}
\scntext{примечание}{Множество \textit{ответов на команды на SC-JSON-коде} легко расширяемо вместе с расширением \textit{команд на SC-JSON-коде}.}

\scnheader{команда создания sc-элементов}
\scnidtf{create elements command}
\scnsubset{команда на SC-JSON-коде}
\scnrelfrom{пример}{Пример команды создания sc-элементов}
\begin{scnindent}
    \scneqimage[12em]{sc-server/images/create_elements_command_example.png}
    \scniselement{команда создания sc-элементов}
    \scnrelfrom{ответ}{Пример ответа на команду создания sc-элементов}
    \scntext{интерпретация}{Обработать команду создания sc-элементов: sc-узла с типом 1 (неуточняемого типа), файла ostis-системы с типом 2 (неуточняемого типа) и содержимым в виде числа с плавающей точкой 45.4 и sc-дуги типа 32 (константного типа) между sc-элементом, находящимся на нулевой позиции в массиве создаваемых sc-элементов, и sc-элементом, находящимся на первой позиции в том же самом массиве.}
\end{scnindent}
\scnrelfrom{класс команд}{ответ на команду создания sc-элементов}
\scntext{примечание}{Стоит отметить, что на уровне интерфейса sc-памяти команда интерпретируется быстро за счёт того, что не используются шаблоны создания изоморфных им конструкций. Также содержимое сообщения \textit{команды создания sc-элементов} может быть пустым.}

\scnheader{ответ на команду создания sc-элементов}
\scnidtf{create elements command answer}
\scnsubset{ответ на команду на SC-JSON-коде}
\scnrelfrom{пример}{Пример ответа на команду создания sc-элементов}
\begin{scnindent}
    \scneqimage[6em]{sc-server/images/create_elements_command_answer_example.png}
    \scniselement{ответ на команду создания sc-элементов}
    \scntext{интерпретация}{Созданы sc-элементы с хэшами 323, 534 и 342 соответственно. Команда обработана успешно.}
\end{scnindent}

\scnheader{команда получения соответствующих типов sc-элементов}
\scnidtf{check elements command}
\scnsubset{команда на SC-JSON-коде}
\scnrelfrom{пример}{Пример команды получения соответствующих типов sc-элементов}
\begin{scnindent}
    \scneqimage[12em]{sc-server/images/check_elements_command_example.png}
    \scniselement{команда получения соответствующих типов sc-элементов}
    \scnrelfrom{ответ}{Пример ответа на команду получения соответствующих типов sc-элементов}
    \scntext{интерпретация}{Получить синтаксические типы sc-элементов с хэшами 885 и 1025.}
\end{scnindent}
\scnrelfrom{класс команд}{ответ на команду получения соответствующих типов sc-элементов}
\scntext{примечание}{Содержимое сообщения \textit{команды получения соответствующих типов sc-элементов} может быть пустым.}

\scnheader{ответ на команду получения соответствующих типов sc-элементов}
\scnidtf{check elements command answer}
\scnsubset{ответ на команду на SC-JSON-коде}
\scnrelfrom{пример}{Пример ответа на команду получения соответствующих типов sc-элементов}
\begin{scnindent}
    \scneqimage[8em]{sc-server/images/check_elements_command_answer_example.png}
    \scniselement{ответ на команду получения соответствующих типов sc-элементов}
    \scntext{интерпретация}{Типы sc-элементов 32 и 0 соответственно. Команда обработана успешно.}
\end{scnindent}
\scntext{примечание}{Если sc-элемент с указанным хэшем не существует, то его тип будет равен 0.}

\scnheader{команда удаления sc-элементов}
\scnidtf{delete elements command}
\scnsubset{команда на SC-JSON-коде}
\scnrelfrom{пример}{Пример команды удаления sc-элементов}
\begin{scnindent}
    \scneqimage[12em]{sc-server/images/delete_elements_command_example.png}
    \scniselement{команда удаления sc-элементов}
    \scnrelfrom{ответ}{Пример ответа на команду удаления sc-элементов}
    \scntext{интерпретация}{Удалить sc-элементы с хэшами 885 и 1025.}
\end{scnindent}
\scnrelfrom{класс команд}{ответ на команду удаления sc-элементов}
\scntext{примечание}{Содержимое сообщения \textit{команды удаления sc-элементов} может быть пустым.}

\scnheader{ответ на команду удаления sc-элементов}
\scnidtf{delete elements command answer}
\scnsubset{ответ на команду на SC-JSON-коде}
\scnrelfrom{пример}{Пример ответа на команду удаления sc-элементов}
\begin{scnindent}
    \scneqimage[9em]{sc-server/images/delete_elements_command_answer_example.png}
    \scniselement{ответ на команду удаления sc-элементов}
    \scntext{интерпретация}{Sc-элементы были удалены из sc-памяти успешно.}
\end{scnindent}
\scntext{примечание}{Если sc-элемент с указанным хэшем не существует, ответ на команду будет безуспешным.}

\scnheader{команда обработки ключевых sc-элементов}
\scnidtf{handle keynodes command}
\scnsubset{команда на SC-JSON-коде}
\scnrelfrom{пример}{Пример команды обработки ключевых sc-элементов}
\begin{scnindent}
    \scneqimage[15em]{sc-server/images/handle_keynodes_command_example.png}
    \scniselement{команда обработки ключевых sc-элементов}
    \scnrelfrom{ответ}{Пример ответа на команду обработки ключевых sc-элементов}
    \scntext{интерпретация}{(1) Найти sc-элемент по системному идентификатору \scnqq{any\_system\_identifier}; (2) Разрешить sc-элемент с типом 1 (неуточняемого типа) по системному идентификатору \scnqq{any\_system\_identifier}.}
\end{scnindent}
\scnrelfrom{класс команд}{ответ на команду обработки ключевых sc-элементов}
\scntext{примечание}{Данный класс команд позволяет быстро обращаться к sc-элементам по их системным идентификаторам, поскольку ключевые sc-элементы кэшируются на уровне интерфейса.}

\scnheader{ответ на команду обработки ключевых sc-элементов}
\scnidtf{handle keynodes command answer}
\scnsubset{ответ на команду на SC-JSON-коде}
\scnrelfrom{пример}{Пример ответа на команду обработки ключевых sc-элементов}
\begin{scnindent}
    \scneqimage[10em]{sc-server/images/handle_keynodes_command_answer_example.png}
    \scniselement{ответ на команду обработки ключевых sc-элементов}
    \scntext{интерпретация}{Ключевый sc-элемент с системным идентификатором \scnqq{any\_system\_identifier} не был найден, поэтому был создан. хэш нового ключевого sc-элемента - 128. Команда выполнена успешно.}
\end{scnindent}

\scnheader{команда обработки содержимого файлов ostis-системы}
\scnidtf{handle link contents command}
\scnsubset{команда на SC-JSON-коде}
\scnrelfrom{пример}{Пример команды обработки содержимого файлов ostis-системы}
\begin{scnindent}
    \scneqimage[12em]{sc-server/images/handle_link_contents_command_example.png}
    \scniselement{команда обработки содержимого файлов ostis-системы}
    \scnrelfrom{ответ}{Пример ответа на команду обработки содержимого файлов ostis-системы}
    \scntext{интерпретация}{(1) Установить содержимое 67 типа \scnqq{int} в файл ostis-системы с хэшем 3123; (2) Получить содержимое файла ostis-системы с хэшем 232; (3) Найти файлы ostis-системы с содержимым \scnqq{exist}.}
\end{scnindent}
\scnrelfrom{класс команд}{ответ на команду обработки содержимого файлов ostis-системы}
\scntext{примечание}{Стоит отметить, что в случае, если файл ostis-системы уже имеет содержимое, то при установке нового содержимого старое содержимое будет удалено из памяти. Содержимое файла ostis-системы может быть установлено как пустое.}

\scnheader{ответ на команду обработки содержимого файлов ostis-системы}
\scnidtf{handle link contents command answer}
\scnsubset{ответ на команду на SC-JSON-коде}
\scnrelfrom{пример}{Пример ответа на команду обработки содержимого файлов ostis-системы}
\begin{scnindent}
    \scneqimage[12em]{sc-server/images/handle_link_contents_command_answer_example.png}
    \scniselement{ответ на команду обработки содержимого файлов ostis-системы}
    \scntext{интерпретация}{(1) Содержимое 67 типа \scnqq{int} было установлено успешно в файл ostis-системы с хэшем 3123; (2) Содержимое файла ostis-системы с хэшем 232 - число 67 целочисленного типа; (3) Файлы ostis-системы с содержимым \scnqq{exist}: 324 и 423.}
\end{scnindent}

\scnheader{команда поиска sc-конструкций, изоморфных заданному sc-шаблону}
\scnidtf{search template command}
\scnsubset{команда на SC-JSON-коде}
\scnrelfrom{пример}{Пример команды поиска sc-конструкций, изоморфных заданному sc-шаблону}
\begin{scnindent}
    \scneqimage[15em]{sc-server/images/search_template_command_example.png}
    \scniselement{команда поиска sc-конструкций, изоморфных заданному sc-шаблону}
    \scnrelfrom{ответ}{Пример ответа на команду поиска sc-конструкций, изоморфных заданному sc-шаблону}
    \scntext{интерпретация}{Найти все такие тройки, в которых первым элементом является sc-элемент c хэшем 23123, третьим sc-элементом является файл ostis-системы неуточняемого константного типа c псевдонимом \scnqq{\_trg}, а вторым элементом - sc-дуга типа sc\_edge\_d\_common c псевдонимом \scnqq{\_edge1}, исходящая от sc-элемента c хэшем 23123 и входящая в файл ostis-системы с псевдонимом \scnqq{\_trg}, и найти все такие тройки, в которых первым элементом является sc-элемент c хэшем 231342, третьим элементов является sc-дуга под псевдонимом \scnqq{\_edge1}, а вторым элементом - sc-дуга типа sc\_edge\_access\_const\_pos\_perm, исходящая от sc-элемента c хэшем 231342 и входящая в sc-дугу \scnqq{\_edge1}. На место переменной с псевдонимом \scnqq{\_trg} подставить sc-элемент с хэшем 564.}
\end{scnindent}
\newpage
\scnrelfrom{класс команд}{ответ на команду поиска sc-конструкций, изоморфных заданному sc-шаблону}
\scntext{примечание}{Поиск sс-конструкций по сформированному sc-шаблону осуществляется специализированным модулем, являющимся частью sc-памяти.}

\scnheader{ответ на команду поиска sc-конструкций, изоморфных заданному sc-шаблону}
\scnidtf{search template command answer}
\scnsubset{ответ на команду на SC-JSON-коде}
\scnrelfrom{пример}{Пример ответа на команду поиска sc-конструкций, изоморфных заданному sc-шаблону}
\begin{scnindent}
    \scneqimage[24em]{sc-server/images/search_template_command_answer_example.png}
    \scniselement{ответ на команду поиска sc-конструкций, изоморфных заданному sc-шаблону}
    \scntext{интерпретация}{Найдена одна sc-конструкция, состоящая из двух троек. хэши sc-элементов в тройках: 23123, 4953, 564 и 231342, 533, 4953 соответственно их расположению в тройках. Команда выполнена успешно.}
    \end{scnindent}
\scntext{примечание}{Важно отметить, что sc-шаблон поиска sc-конструкций не может быть пустым.}

\scnheader{команда создания sc-конструкции, изоморфной заданному sc-шаблону}
\scnidtf{generate template command}
\scnsubset{команда на SC-JSON-коде}
\scnrelfrom{пример}{Пример команды создания sc-конструкции, изоморфной заданному sc-шаблону}
\begin{scnindent}
    \scneqimage[15em]{sc-server/images/generate_template_command_example.png}
    \scniselement{команда создания sc-конструкции, изоморфной заданному sc-шаблону}
    \scnrelfrom{ответ}{Пример ответа на команду создания sc-конструкции, изоморфной заданному sc-шаблону}
    \scntext{интерпретация}{Создать такую тройку, в которой первым элементом является sc-элемент c хэшем 589, третьим sc-элементом является sc-узел неуточняемого типа c псевдонимом \scnqq{\_trg}, а вторым элементом - sc-дуга типа sc\_edge\_d\_common c псевдонимом \scnqq{\_edge1}, исходящая от sc-элемента c хэшем 589 и входящая в sc-узел с псевдонимом \scnqq{\_trg}. На место переменной с псевдонимом \scnqq{\_trg} подставить sc-элемент с хэшем 332.}
\end{scnindent}
\scnrelfrom{класс команд}{ответ на команду создания sc-конструкции, изоморфной заданному sc-шаблону}
\scntext{примечание}{Создание sс-конструкции по сформированному sc-шаблону осуществляется специализированным модулем, являющимся частью sc-памяти.}

\scnheader{ответ на команду создания sc-конструкции, изоморфной заданному sc-шаблону}
\scnidtf{search template command answer}
\scnsubset{ответ на команду на SC-JSON-коде}
\scnrelfrom{пример}{Пример ответа на команду создания sc-конструкции, изоморфной заданному sc-шаблону}
\begin{scnindent}
    \scneqimage[15em]{sc-server/images/generate_template_command_answer_example.png}
    \scniselement{ответ на команду создания sc-конструкции, изоморфной заданному sc-шаблону}
    \scntext{интерпретация}{Создана одна sc-конструкция, состоящая из одной тройки. хэши sc-элементов в тройке: 128, 589, 332 соответственно их расположению в тройках. Команда выполнена успешно.}
\end{scnindent}
\scntext{примечание}{Важно отметить, что sc-шаблон создания sc-конструкции не может быть пустым.}

\scnheader{команда обработки sc-событий}
\scnidtf{handle events command}
\scnsubset{команда на SC-JSON-коде}
\scnrelfrom{пример}{Пример команды обработки sc-событий}
\begin{scnindent}
    \scneqimage[15em]{sc-server/images/handle_events_command_example.png}
    \scniselement{команда обработки sc-событий}
    \scnrelfrom{ответ}{Пример ответа на команду обработки sc-событий}
    \scntext{интерпретация}{(1) Зарегистрировать sc-событие типа \scnqq{add\_outgoing\_edge} по sc-элементу с хэшем 324; (2) Разрегистрировать sc-события с позициями sc-элементов 2, 4 и 5 соответственно.}
\end{scnindent}
\scnrelfrom{класс команд}{ответ на команду обработки sc-событий}
\scnrelfrom{класс команд}{ответ инициализации sc-события}
\scntext{примечание}{\textit{Ответ инициализации sc-события} может производиться несколько раз за разные промежутки времени.}

\scnheader{ответ на команду обработки sc-событий}
\scnidtf{handle events command answer}
\scnsubset{ответ на команду на SC-JSON-коде}
\scnsuperset{SC-JSON-код}
\scnrelfrom{пример}{Пример ответа на команду обработки sc-событий}
\begin{scnindent}
    \scneqimage[9em]{sc-server/images/handle_events_command_answer_example.png}
    \scniselement{ответ на команду обработки sc-событий}
    \scntext{интерпретация}{(1) Sc-событие типа \scnqq{add\_outgoing\_edge} по sc-элементу с хэшем 324 было зарегистрировано успешно на 7-ой позиции очереди sc-событий; (2) Sc-события под позициями 2, 4, 5 удалены успешно.}
\end{scnindent}

\scnheader{ответ инициализации sc-события}
\scnidtf{init event command answer}
\scnsubset{ответ на команду на SC-JSON-коде}
\scnrelfrom{пример}{Пример ответа инициализации sc-события}
\begin{scnindent}
    \scneqimage[9em]{sc-server/images/init_event_command_answer_example.png}
    \scniselement{ответ инициализации sc-события}
    \scntext{интерпретация}{Sc-событие было инициализировано успешно: добавлена выходящая sc-дуга с хэшем 328 из зарегистрированного sc-элемента с хэшем 324 в sc-элемент c хэшем 35. Статус sc-события - 1.}
\end{scnindent}

\end{scnsubstruct}
\scnsourcecomment{Завершили представление \textit{Синтаксиса SC-JSON-кода}}

\scnheader{Серверная система на основе Websocket, обеспечивающая доступ к sc-памяти платформы интерпретации sc-моделей при помощи команд SC-JSON-кода}
\scnidtf{Система, работающая по принципам Websocket и предоставляющая параллельно-асинхронный многоклиентский доступ к sc-памяти платформы интерпретации sc-моделей при помощи SC-JSON-кода}
\scnidtf{SC-JSON-сервер}
\scntext{часто используемый неосновной внешний идентификатор sc-элемента}{SC-сервер}
\scniselement{многократно используемый компонент ostis-систем}
\scniselement{атомарный многократно используемый компонент ostis-систем}
\scniselement{зависимый многократно используемый компонент ostis-систем}
\scnrelto{компонент системы}{Программный вариант реализации платформы интерпретации sc-моделей компьютерных систем}
\begin{scnrelfromlist}{автор}
	\scnitem{Зотов Н. В.}
\end{scnrelfromlist}
\begin{scnrelfromlist}{используемый язык программирования}
	\scnitem{С}
	\scnitem{C++}
\end{scnrelfromlist}
\begin{scnrelfromlist}{используемый язык}
	\scnitem{SC-JSON-код}
\end{scnrelfromlist}
\begin{scnrelfromlist}{зависимости компонента}
    \scnitem{Библиотека программных компонентов для обработки и, задающее синтаксис json-текстов JSON for Modern C++ версии 3.10.5}
    \begin{scnindent}
	    \scnidtf{nlohmann-json 3.10.5}
	    \scnrelto{версия компонента}{Библиотека программных компонентов для обработки и, задающее синтаксис json-текстов JSON for Modern C++}
	    \begin{scnindent}
		    \scnidtf{nlohmann-json}
		    \scniselement{многократно используемый компонент ostis-систем}
		    \scniselement{неатомарный многократно используемый компонент ostis-систем}
		    \scniselement{зависимый многократно используемый компонент ostis-систем}
		    \scntext{адрес хранилища}{https://github.com/nlohmann/json}
		    \begin{scnindent}
			    \scniselement{адрес хранилища на GitHub}
		    \end{scnindent}
		    \scntext{скрипт установки}{
			    sudo add-apt-repository universe\\
			    sudo apt-get update\\
			    sudo apt-get install -y nlohmann-json3-dev
		    }
		    \begin{scnindent}
			    \scniselement{скрипт на языке Bash}
			    \scniselement{скрипт на языке Bash, поддерживаемый семейством операционных систем UNIX}
		    \end{scnindent}
		    \scntext{скрипт установки}{
			    brew install nlohmann-json
		    }
		    \begin{scnindent}
			    \scniselement{скрипт на языке Bash}
			    \scniselement{скрипт на языке Bash, поддерживаемый семейством операционных систем MaсOS}
		    \end{scnindent}
	    \end{scnindent}
    \end{scnindent}
    \scnitem{Библиотека кросс-платформенных программных компонентов для реализации серверных приложений на основе Websocket WebSocket++ версии 0.8.2}
    \begin{scnindent}
    	\scnidtf{websocketcpp 0.8.2}
    	\scnrelto{версия компонента}{Библиотека кросс-платформенных программных компонентов для реализации серверных приложений на основе Websocket WebSocket++}
    	\begin{scnindent}
    		\scnidtf{websocketcpp}
    		\scniselement{многократно используемый компонент ostis-систем}
    		\scniselement{неатомарный многократно используемый компонент ostis-систем}
    		\scniselement{зависимый многократно используемый компонент ostis-систем}
    		\scntext{адрес хранилища}{https://github.com/zaphoyd/websocketpp}
    		\begin{scnindent}
    			\scniselement{адрес хранилища на GitHub}
    		\end{scnindent}
    		\scntext{скрипт установки}{
    			sudo apt-get update\\
    			sudo apt-get install -y libwebsocketpp-dev
    		}
    		\begin{scnindent}
    			\scniselement{скрипт на языке Bash}
    			\scniselement{скрипт на языке Bash, поддерживаемый семейством операционных систем UNIX}
    		\end{scnindent}
    		\scntext{скрипт установки}{
    			brew install libwebsocketpp
    		}
    		\begin{scnindent}
    			\scniselement{скрипт на языке Bash}
    			\scniselement{скрипт на языке Bash, поддерживаемый семейством операционных систем MaсOS}
    		\end{scnindent}
    	\end{scnindent}
    \end{scnindent}
    \scnitem{Программный компонент настройки программных компонентов ostis-систем версии 0.1.0}
    \begin{scnindent}
    	\scnidtf{sc-config-utils 0.1.0}
    	\scnrelto{версия компонента}{Программный компонент настройки программных компонентов ostis-систем}
    	\begin{scnindent}
    		\scnidtf{sc-config-utils}
    		\scniselement{многократно используемый компонент ostis-систем}
    		\scniselement{неатомарный многократно используемый компонент ostis-систем}
    		\scniselement{зависимый многократно используемый компонент ostis-систем}
    		\begin{scnrelfromlist}{автор}
    		    \scnitem{Зотов Н. В.}
    		    \scnitem{Насевич П. Е.}
    		    \scnitem{Хорошавин В. Д.}
    		\end{scnrelfromlist}
    		\scntext{адрес хранилища}{https://github.com/ostis-ai/sc-machine/tree/main/sc-tools/sc-config-utils}
    		\begin{scnindent}
    			\scniselement{адрес хранилища на GitHub}
    		\end{scnindent}
    	\end{scnindent}
    \end{scnindent}
    \scnitem{Программная модель sc-памяти версии 0.6.1}
    \scnidtf{sc-machine 0.6.1}
    \begin{scnindent}
    	\scnrelto{версия компонента}{Программная модель sc-памяти}
    \end{scnindent}
\end{scnrelfromlist}
\scntext{адрес хранилища}{https://github.com/ostis-ai/sc-machine/tree/main/sc-tools/sc-server}
\begin{scnindent}
	\scniselement{адрес хранилища на GitHub}
\end{scnindent}
\scntext{пояснение}{\textit{Серверная система на основе Websocket, обеспечивающая доступ к sc-памяти платформы интерпретации sc-моделей при помощи команд SC-JSON-кода}, представляет собой интерпретатор команд и ответов на них \textit{SC-JSON-кода} на программное представление sc-конструкций в sc-памяти при помощи Библиотеки программных компонентов для обработки и, задающее синтаксис json-текстов JSON for Modern C++ и Библиотека кросс-платформенных программных компонентов для реализации серверных приложений на основе Websocket WebSocket++, а также обеспечивается комплексным тестовым покрытием посредством программных фреймворков Google Tests и Google Benchmark Tests. Библиотека программных компонентов для обработки и, задающее синтаксис json-текстов JSON for Modern C++ имеет богатый, удобный и быстродействующий функционал, необходимый для реализации подобных компонентов ostis-систем, а Библиотеки кросс-платформенных программных компонентов для реализации серверных приложений на основе Websocket WebSocket++ позволяет элегантно проектировать серверные системы без использовании избыточных зависимостей и решение. Настройка программного компонента осуществляется с помощью \textit{Программного компонента настройки программных компонентов ostis-систем} и скриптов языков CMake и Bash.}
\scntext{пояснение}{Стоит отметить, что текущая реализация \textit{Серверной системы на основе Websocket, обеспечивающая доступ к sc-памяти платформы интерпретации sc-моделей при помощи команд SC-JSON-кода} не является первой в своём роде и заменяет предыдущую её реализацию, написанную на языке Python. Причиной такой замены состоит в следующем:
	\begin{scnitemize}
		\item предыдущая реализация \textit{Серверной системы на основе Websocket, обеспечивающая доступ к sc-памяти платформы интерпретации sc-моделей при помощи команд SC-JSON-кода}, реализованная на языке программирования Python, зависит от библиотеки Boost Python, предоставляемой сообществом по развитию и коллаборации языков С++ и Python. Дело в том, что такое решение требует поддержки механизма интерпретации программного кода на языке Python на язык С++, что является избыточным и необоснованным, поскольку большая часть программного кода \textit{\textbf{Программного варианта реализации платформы интерпретации sc-моделей компьютерных систем}} реализована на языках С и С++. Новая реализация описываемой программной системы позволяет избавиться от использования ёмких и ресурсозатратных библиотек (например, boost-python-lib, llvm) и языка Python;
		\item при реализации распределённых подсистем важную роль играет скорость обработки знаний, то есть возможность быстро и срочно отвечать на запросы пользователя. Качество доступа к sc-памяти посредством реализованной \textit{Подсистемы взаимодействия с sc-памятью на основе языка JSON} должно быть соизмеримо с качеством доступа к sc-памяти при помощи специализированного программного интерфейса API, реализованного на том же языке программирования, что и сама система. Новая реализация позволяет повысить скорость обработки запросов \textit{Подсистемой взаимодействия с sc-памятью на основе языка JSON}, в том числе и обработка знаний, не менее чем в 1,5 раза по сравнению с предыдущим вариантом реализации этой подсистемы.
	\end{scnitemize}
}
\scnrelfrom{формальная модель}{Модель \textit{Серверной системы на основе Websocket, обеспечивающая доступ к sc-памяти платформы интерпретации sc-моделей при помощи команд SC-JSON-кода}}
\begin{scnrelfromlist}{характеристика}
	\scnitem{\scnfilelong{\textit{Серверная система на основе Websocket, обеспечивающая доступ к sc-памяти платформы интерпретации sc-моделей при помощи команд SC-JSON-кода} обеспечивает многофункциональный доступ к sc-памяти ostis-системы и удовлетворяет требованиям своей модели. С точки зрения прагматики, программный компонент имеет такой же специализированный программный интерфейс, как и \textit{Программная модель sc-памяти}, однако взаимодействие с ним осуществляется посредством сети.}}
	\scnitem{\scnfilelong{Реализованный программный компонент позволяет многопользовательский асинхронный доступ к sc-памяти. В ходе тестирования sc-сервера выяснилось, что его реализация позволяет обрабатывать запросы 1000 клиентских систем. В связи с необходимостью обеспечения параллельного доступа к sc-памяти на уровне реализации программного компонента были добавлены блоки синхронизации. Среди таких можно заметить очередь команд на обработку системой - вне зависимости от того сколько клиентских систем и в каком количестве они отправляют команды на обработку, все команды добавляются в очередь - такое решение позволит обойти проблемы взаимодействия блоков синхронизации на уровне sc-памяти. При этом серверную систему невозможно отключить до тех пор, пока очередь команд имеет какие-нибудь элементы. Также серверная система продолжает работать, если в списке идентификаторов клиентских систем остались некоторые идентификаторы этих клиентских систем. Эти функции обуславливаются необходимостью поддержки атомарности запросов, обрабатываемых системой.}}
	\scnitem{\scnfilelong{В процессе тестирования реализации \textit{Серверной системы на основе Websocket, обеспечивающая доступ к sc-памяти платформы интерпретации sc-моделей при помощи команд SC-JSON-кода}, были получены оценки скорости обработки запросов этим компонентом. При нагрузочном тестировании использовалась тестовая клиентская система, написанная на С++ и не имеющая функционала обработки текстов SC-JSON-кода. В результате тестирования было выяснено, что при отправке серверной системы 1000 различных команд: команд создания sc-элементов, команд обработки содержимого файлов ostis-системы и команд удаления sc-элементов, время потраченное на их обработку не превышало 0,2 секунды. При этом в отдельных случаях на обработку 1000 команд создания sc-элементов уходило не более 0,14 секунды, команд удаления sc-элементов - не более 0,07 секунды, команд обработки содержимого файлов ostis-системы - не более 0,27 секунды, команд поиска sc-конструкций, изоморфных заданному sc-шаблону - не более 0,45 секунды.}}
\end{scnrelfromlist}

\scnheader{Модель \textit{Серверной системы на основе Websocket, обеспечивающая доступ к sc-памяти платформы интерпретации sc-моделей при помощи команд SC-JSON-кода}}
\begin{scnreltovector}{декартово произведение}
	\scnitem{\scnnonamednode}
	\begin{scnindent}
		\begin{scnreltovector}{декартово произведение}
			\scnitem{\scnnonamednode}
			\begin{scnindent}
				\begin{scnreltovector}{декартово произведение}
					\scnitem{\scnnonamednode}
					\begin{scnindent}
    					\begin{scnreltovector}{декартово произведение}
    						\scnitem{функция создания sc-элементов по их описаниям: типам и инцидентным sc-элементам*}
    						\scnitem{функция получения соответствующих синтаксических типов sc-элементов*}
    					\end{scnreltovector}
					\end{scnindent}
					\scnitem{\scnnonamednode}
					\begin{scnindent}
    				    \begin{scnreltovector}{декартово произведение}
    						\scnitem{функция проверки существования sc-элементов в sc-памяти*}
    						\scnitem{функция получения ключевого sc-элемента по системному идентификатору*}
    					\end{scnreltovector}
					\end{scnindent}
				\end{scnreltovector}
			\end{scnindent}
			\scnitem{\scnnonamednode}
			\begin{scnindent}
				\begin{scnreltovector}{декартово произведение}
					\scnitem{\scnnonamednode}
					\begin{scnindent}
    					\begin{scnreltovector}{декартово произведение}
    						\scnitem{функция разрешения ключевого sc-элемента по системному идентификатору*}
    						\scnitem{функция установки содержимого в файл ostis-системы*}
    					\end{scnreltovector}
					\end{scnindent}
					\scnitem{\scnnonamednode}
					\begin{scnindent}
    					\begin{scnreltovector}{декартово произведение}
    						\scnitem{функция получения содержимого из файлов ostis-системы*}
    						\scnitem{функция получения файлов ostis-системы по содержимому*}
    					\end{scnreltovector}
					\end{scnindent}
				\end{scnreltovector}
			\end{scnindent}
		\end{scnreltovector}
	\end{scnindent}
	\scnitem{\scnnonamednode}
	\begin{scnindent}
		\begin{scnreltovector}{декартово произведение}
			\scnitem{\scnnonamednode}
			\begin{scnindent}
				\begin{scnreltovector}{декартово произведение}
					\scnitem{функция поиска sc-конструкций, изоморфных заданному sc-шаблону, состоящего из троек, специфицированных инцидентными sc-элементами и/или синтаксическими типами и/или идентификаторами и/или псевдонимами их sc-элементов*}
					\scnitem{функция создания sc-конструкции, изоморфной заданному sc-шаблону, состоящего из троек, специфицированных инцидентными sc-элементами и/или синтаксическими типами и/или идентификаторами и/или псевдонимами их sc-элементов*}
				\end{scnreltovector}
			\end{scnindent}
			\scnitem{\scnnonamednode}
			\begin{scnindent}
				\begin{scnreltovector}{декартово произведение}
					\scnitem{функция подписки sc-события на sc-элемент*}
					\scnitem{функция отмены подписки sc-события на sc-элемент*}
				\end{scnreltovector}
			\end{scnindent}
		\end{scnreltovector}
	\end{scnindent}
\end{scnreltovector}
\scntext{пояснение}{\textit{Модель Серверной системы на основе Websocket, обеспечивающая доступ к sc-памяти платформы интерпретации sc-моделей при помощи команд SC-JSON-кода} описывает необходимы и достаточный программный интерфейс для взаимодействия c sc-памятью. В общем случае описывает функциональные возможности не только \textit{Серверной системы на основе Websocket, обеспечивающая доступ к sc-памяти платформы интерпретации sc-моделей при помощи команд SC-JSON-кода}, но клиентских систем взаимодействующих с ней, поскольку зачастую эти клиентские системы включают специализированный программный интерфейс, схожий с интерфейсом серверной системы, но на другом языке программирования.}
\begin{scneqtoset}
	\scnitem{функция создания sc-элементов по их описаниям: типам и инцидентным sc-элементам*}
	\begin{scnindent}
		\begin{scnreltovector}{декартово произведение}
			\scnitem{\scnnonamednode}
			\begin{scnindent}
			    \begin{scnreltovector}{декартово произведение}
				    \scnitem{Множество всех синтаксических типов sc-элементов}
				    \scnitem{Множество всех sc-элементов, имеющих инцидентные sc-элементы в sc-памяти}
			    \end{scnreltovector}
			\end{scnindent}
			\scnitem{Множество всех sc-элементов, хранящихся в sc-памяти}
		\end{scnreltovector}
		\scntext{примечание}{Создается sc-элемент заданного синтаксического типа из Множества всех синтаксических типов sc-элементов и имеющий заданные связи с заданными элементами из Множества всех sc-элементов, имеющих инцидентные sc-элементы в sc-памяти, дополняя Множество всех sc-элементов, хранящихся в sc-памяти.}
	\end{scnindent}
	\scnitem{функция получения соответствующих синтаксических типов sc-элементов*}
	\begin{scnindent}
		\begin{scnreltovector}{декартово произведение}
			\scnitem{Множество всех sc-элементов, хранящихся в sc-памяти}
			\scnitem{Множество всех синтаксических типов sc-элементов}
		\end{scnreltovector}
		\scntext{примечание}{У каждого sc-элемента из Множества всех sc-элементов, хранящихся в sc-памяти, можно получить соответствующий синтаксический тип из Множества всех синтаксических типов sc-элементов.}
	\end{scnindent}
	\scnitem{функция проверки существования sc-элементов в sc-памяти*}
	\begin{scnindent}
		\begin{scnreltovector}{декартово произведение}
			\scnitem{Множество всех sc-элементов, хранящихся в sc-памяти}
			\scnitem{Множество, состоящее из знаков истины и лжи}
		\end{scnreltovector}
		\scntext{примечание}{Каждый sc-элемент из Множества всех sc-элементов, хранящихся в sc-памяти, должен находиться в sc-памяти.}
	\end{scnindent}
	\scnitem{функция получения ключевого sc-элемента по системному идентификатору*}
	\begin{scnindent}
		\begin{scnreltovector}{разность множеств}
			\scnitem{\scnnonamednode}
			\begin{scnindent}
				\begin{scnreltoset}{объединение}
					\scnitem{\scnnonamednode}
					\begin{scnindent}
    					\begin{scnreltovector}{декартово произведение}
    						\scnitem{Множество всех sc-элементов, хранящихся в sc-памяти}
    						\scnitem{Множество всех ключевых sc-элементов, хранящихся в sc-памяти}
    					\end{scnreltovector}
					\end{scnindent}
				    \scnitem{\scnnonamednode}
					\begin{scnindent}
					    \begin{scnreltovector}{декартово произведение}
						    \scnitem{Множество всех системных идентификаторов sc-элементов, хранящихся в файловом хранилище}
						    \scnitem{Множество всех ключевых sc-элементов, хранящихся в sc-памяти}
					    \end{scnreltovector}
					\end{scnindent}
				\end{scnreltoset}
			\end{scnindent}
			\scnitem{функция разрешения ключевого sc-элемента по системному идентификатору*}
		\end{scnreltovector}
		\scntext{примечание}{Каждый ключевой sc-элемент из Множества всех ключевых sc-элементов, хранящихся в sc-памяти, является sc-элементом Множества всех sc-элементов, хранящихся в sc-памяти, и имеет, по крайней мере, системный идентификатор из Множества всех системных идентификаторов sc-элементов, хранящихся в файловом хранилищеs.}
	\end{scnindent}
	\scnitem{функция разрешения ключевого sc-элемента по системному идентификатору*}
	\begin{scnindent}
		\begin{scnreltovector}{разность множеств}
			\scnitem{\scnnonamednode}
			\begin{scnindent}
    			\begin{scnreltoset}{объединение}
    				\scnitem{\scnnonamednode}
    				\begin{scnindent}
        				\begin{scnreltovector}{декартово произведение}
        					\scnitem{Множество всех sc-элементов, хранящихся в sc-памяти}
        					\scnitem{Множество всех ключевых sc-элементов, хранящихся в sc-памяти}
        				\end{scnreltovector}
    				\end{scnindent}
    				\scnitem{\scnnonamednode}
    				\begin{scnindent}
        				\begin{scnreltovector}{декартово произведение}
        					\scnitem{Множество всех системных идентификаторов sc-элементов, хранящихся в файловом хранилище}
        					\scnitem{Множество всех ключевых sc-элементов, хранящихся в sc-памяти}
        				\end{scnreltovector}
    				\end{scnindent}
    			\end{scnreltoset}
			\end{scnindent}
			\scnitem{функция получения ключевого sc-элемента по системному идентификатору*}
		\end{scnreltovector}
		\scntext{примечание}{Из каждого sc-элемента Множества всех sc-элементов, хранящихся в sc-памяти, можно получить ключевой sc-элемент Множества всех ключевых sc-элементов, хранящихся в sc-памяти, зная, по крайней мере, его системный идентификатор из Множества всех системных идентификаторов sc-элементов, хранящихся в файловом хранилище.}
	\end{scnindent}
	\scnitem{функция установки содержимого в файл ostis-системы*}
	\begin{scnindent}
		\begin{scnreltovector}{декартово произведение}
			\scnitem{\scnnonamednode}
			\begin{scnindent}
    			\begin{scnreltovector}{декартово произведение}
    				\scnitem{Множество всех файлов ostis-системы, хранящихся в sc-памяти}
    				\scnitem{Множество внешних знаков, являющихся содержимым файлов ostis-системы, хранящиеся в файловом хранилище}
    			\end{scnreltovector}
			\end{scnindent}
			\scnitem{Множество, состоящее из знаков истины и лжи}
		\end{scnreltovector}
		\scntext{примечание}{В каждый sc-элемент из Множества всех файлов ostis-системы, хранящихся в sc-памяти может быть установлено содержимое из Множества всего содержимого файлов ostis-системы, хранящегося в файловом хранилище.}
	\end{scnindent};
	\scnitem{функция получения содержимого из файлов ostis-системы*}
	\begin{scnindent}
		\begin{scnreltovector}{декартово произведение}
			\scnitem{Множество всех файлов ostis-системы, хранящихся в sc-памяти}
			\scnitem{Множество внешних знаков, являющихся содержимым файлов ostis-системы, хранящиеся в файловом хранилище}
		\end{scnreltovector}
		\scntext{примечание}{Из каждого файла ostis-системы Множества всех файлов ostis-системы, хранящихся в sc-памяти, можно получить содержимое, принадлежащее Множеству внешних знаков, являющихся содержимым файлов ostis-системы, хранящиеся в файловом хранилище.}
	\end{scnindent}
	\scnitem{функция получения файлов ostis-системы по содержимому*}
	\begin{scnindent}
		\begin{scnreltovector}{декартово произведение}
			\scnitem{Множество внешних знаков, являющихся содержимым файлов ostis-системы, хранящиеся в файловом хранилище}
			\scnitem{Множество всех файлов ostis-системы, хранящихся в sc-памяти}
		\end{scnreltovector}
		\scntext{примечание}{По содержимому из Множества внешних знаков, являющихся содержимым файлов ostis-системы, хранящиеся в файловом хранилище, можно получить подмножество файлов ostis-системы из Множества всех файлов ostis-системы, хранящихся в sc-памяти, в которые установлено это содержимое.}
	\end{scnindent}
	\scnitem{функция поиска sc-конструкций, изоморфных заданному sc-шаблону, состоящего из троек, специфицированных инцидентными sc-элементами и/или синтаксическими типами и/или идентификаторами и/или псевдонимами их sc-элементов*}
	\begin{scnindent}
		\begin{scnreltovector}{разность множеств}
			\scnitem{\scnnonamednode}
			\begin{scnindent}
				\begin{scnreltovector}{декартово произведение}
					\scnitem{\scnnonamednode}
					\begin{scnindent}
    					\begin{scnreltovector}{декартово произведение}
    						\scnitem{\scnnonamednode}
    						\begin{scnindent}
        						\begin{scnreltovector}{декартово произведение}
        							\scnitem{Множество всех sc-элементов, хранящихся в sc-памяти}
        							\scnitem{Множество всех синтаксических типов sc-элементов}
        						\end{scnreltovector}
    						\end{scnindent}
    						\scnitem{\scnnonamednode}
    						\begin{scnindent}
        						\begin{scnreltovector}{декартово произведение}
        							\scnitem{Множество всех sc-элементов, имеющих инцидентные sc-элементы в sc-памяти}
        							\scnitem{Множество всех идентификаторов sc-элементов, хранящихся в файловом хранилище}
    						    \end{scnreltovector}
    						\end{scnindent}
    					\end{scnreltovector}
					\end{scnindent}
					\scnitem{Множество всех sc-структур}
				\end{scnreltovector}
			\end{scnindent}
			\scnitem{функция создания sc-конструкции, изоморфной заданному sc-шаблону, состоящего из троек, специфицированных инцидентными sc-элементами и/или синтаксическими типами и/или идентификаторами и/или псевдонимами их sc-элементов*}
		\end{scnreltovector}
		\scntext{примечание}{По синтаксическим типам sc-элементов из Множества всех синтаксических типов sc-элементов и/или самим sc-элементам из Множества всех sc-элементов, хранящихся в sc-памяти, связанными с sc-элементами, у которых известны синтаксические типы из Множества всех синтаксических типов sc-элементов, в том числе инцидентным sc-элементами из Множества всех sc-элементов, имеющих инцидентные sc-элементы в sc-памяти, и/или идентификаторам (в том числе псевдонимам) sc-элементов Множества всех идентификаторов sc-элементов, хранящихся в файловом хранилище можно найти sc-структуры, принадлежащие Множеству всех sc-структур и состоящие из всех этих sc-элементов.}
	\end{scnindent}
	\scnitem{функция создания sc-конструкции, изоморфной заданному sc-шаблону, состоящего из троек, специфицированных инцидентными sc-элементами и/или синтаксическими типами и/или идентификаторами и/или псевдонимами их sc-элементов*}
	\begin{scnindent}
		\begin{scnreltovector}{разность множеств}
			\scnitem{\scnnonamednode}
			\begin{scnindent}
				\begin{scnreltovector}{декартово произведение}
					\scnitem{\scnnonamednode}
					\begin{scnindent}
    					\begin{scnreltovector}{декартово произведение}
    						\scnitem{\scnnonamednode}
    						\begin{scnindent}
        						\begin{scnreltovector}{декартово произведение}
        							\scnitem{Множество всех sc-элементов, хранящихся в sc-памяти}
        							\scnitem{Множество всех синтаксических типов sc-элементов}
        						\end{scnreltovector}
    						\end{scnindent}
    						\scnitem{\scnnonamednode}
    						\begin{scnindent}
    						    \begin{scnreltovector}{декартово произведение}
    							    \scnitem{Множество всех sc-элементов, имеющих инцидентные sc-элементы в sc-памяти}
    							    \scnitem{Множество всех идентификаторов sc-элементов, хранящихся в файловом хранилище}
    						    \end{scnreltovector}
    						\end{scnindent}
    					\end{scnreltovector}
					\end{scnindent}
					\scnitem{Множество всех sc-структур}
				\end{scnreltovector}
			\end{scnindent}
			\scnitem{функция поиска sc-конструкций, изоморфных заданному sc-шаблону, состоящего из троек, специфицированных инцидентными sc-элементами и/или синтаксическими типами и/или идентификаторами и/или псевдонимами их sc-элементов*}
		\end{scnreltovector}
		\scntext{примечание}{По синтаксическим типам sc-элементов из Множества всех синтаксических типов sc-элементов и/или самим sc-элементам из Множества всех sc-элементов, хранящихся в sc-памяти, связанными с sc-элементами, у которых известны синтаксические типы из Множества всех синтаксических типов sc-элементов, в том числе инцидентным sc-элементами из Множества всех sc-элементов, имеющих инцидентные sc-элементы в sc-памяти, и/или идентификаторам (в том числе псевдонимам) sc-элементов Множества всех идентификаторов sc-элементов, хранящихся в файловом хранилище всегда можно создать sc-структуру, дополняющую Множество всех sc-структур и состоящую из всех этих sc-элементов.}
	\end{scnindent}
	\scnitem{функция подписки sc-события на sc-элемент}
	\begin{scnindent}
		\begin{scnreltovector}{декартово произведение}
			\scnitem{\scnnonamednode}
			\begin{scnindent}
    			\begin{scnreltovector}{декартово произведение}
    				\scnitem{Множество всех sc-элементов, храняшихся в sc-памяти}
    				\scnitem{Множество всех возможных типов sc-событий в sc-памяти}
			    \end{scnreltovector}
			\end{scnindent}
			\scnitem{Множество всех sc-событий, зарегистрированных в sc-памяти}
		\end{scnreltovector}
		\scntext{примечание}{На каждый sc-элемент Множества всех sc-элементов, храняшихся в sc-памяти можно подписать sc-cобытие из Множества всех sc-событий, зарегистрированных в sc-памяти типа из Множества всех sc-событий, зарегистрированных в sc-памяти.}
	\end{scnindent}
	\scnitem{функция отмены подписки sc-события на sc-элемент}
	\begin{scnindent}
		\begin{scnreltovector}{декартово произведение}
			\scnitem{\scnnonamednode}
			\begin{scnindent}
    			\begin{scnreltovector}{декартово произведение}
    				\scnitem{Множество всех sc-событий, зарегистрированных в sc-памяти}
    				\scnitem{Множество всех sc-элементов, храняшихся в sc-памяти}
    			 \end{scnreltovector}
			\end{scnindent}
			\scnitem{Множество, состоящее из знаков истины и лжи}
		 \end{scnreltovector}
		\scntext{примечание}{На каждый sc-элемент Множества всех sc-элементов, храняшихся в sc-памяти можно отписать от sc-cобытия из Множества всех sc-событий, зарегистрированных в sc-памяти.}
	\end{scnindent}
\end{scneqtoset}

\end{scnsubstruct}
\scnsourcecomment{Завершили описание \textit{Подсистемы взаимодействия c sc-памятью на основе языка JSON}}


\bigskip
\end{scnsubstruct} \scnendsegmentcomment{\textit{Описание реализации подсистемы взаимодействия с внешней средой с использованием сетевых языков}}
